\chapter{Theoretical motivation}
\label{sec:theory}

The Standard Model (SM) of particles represents, so far, the best available description of the particles and their interactions. It is the summation of two gauge theories: the electroweak interaction, that pictures together the weak and electromagnetic interactions, and the strog interaction, or Quantum Chromodynamics (QCD). Particles, namely quarks and leptons, are described by spin 1/2 fermions, whilst interactions are embodied by spin 1 bosons. The simmetry group of the standard model is:
\begin{equation}
SU_{C}(3) \times SU_L (2) \times U_Y (1),
\end{equation}
\label{eq:theory_SMgroup}
where the first factor is related to strong interactions, whose mediators are eight gluons, whilst $SU_L (2) \times U_Y (1)$ is the electroweak simmetry group, whose mediators are photons and $Z$-$W^{\pm}$ bosons.

In renormalizable theories, with no anomalies, all gauge bosons are expected to be massless, in contrast with our experimental knowledge (cite W-Z discovery). This kind of dilemma can be solved by introducing a new scalar particle, the Higgs boson (cite Higgs article), that can give mass to weak bosons and fermions via the spontaneous symmetry breaking mechanism.

In the last decades, Standard Model has been accurately probed by many experimental facilities (LEP, Tevatron), demonstrating an impressive agreement between theoretical predictions and experimental results. The discovery of the Higgs boson at the CERN Large Hadron Collider, measured by both CMS and ATLAS collaborations~\cite{bib:Aad20121}-\cite{bib:Chatrchyan201230}-\cite{bib:Chatrchyan2013lba}-\cite{Aad:2013xqa}-\cite{Khachatryan:2014jba}-\cite{Aad:2014aba}-\cite{Aad:2015zhl}, represents not only an extraordinary confirmation of the model, but also the latest biggest achievement in particle physics as a whole.

\section{Beyond Standard Model theories}
Even though the Standard Model is the most complete picture of the universe of the particles, many questions are still left open by the model. From a phenomenological point of view, some experimental observations are not included in the theory:
\begin{itemize}
\item in SM, neutrinos are massless (whilst experimentally it has been confirmed to be non-zero, i.e. by the neutrino oscillation);
\item no candidate for the dark matter is foreseen (whilst it has been observed in cosmology);
\item no fields included in the SM can explain the cosmological inflation;
\item SM can not justify the matter-antimatter asymmetry.
\end{itemize}
From a purely theoretical perspective, some issues are still relevant in the formulation of the model:
\begin{itemize}
\item {\itshape Flavour problem.}\\ The Standard Model has 18 free parameters: 9 fermionic masses; 3 angular parameters in Cabibbo-Kobayashi-Maskawa matrix, plus 1 phase parameter; electromagnetic coupling $\alpha$; strong coupling $\alpha_{strong}$;  weak coupling $\alpha_{weak}$; $Z$ mass; the mass of the Higgs boson. Such a huge number of degrees of freedom is considered as weakly predictive.
\item {\itshape Unification.}\\ There is not a ``complete'' unification of strong, weak and electromagnetic interactions, since each one has its own coupling constant, behaving differently at different energy scales; not to mention the fact that gravitational interaction, completely excluded from the SM.
\item {\itshape Hierarchy problem.}\\ From Quantum Field Theory, it is known that perturbative corrections to the mass of the scalar bosons included in the theory tend to make it increase towards the energy scale at which the considered theory is valid [cite n. 4 of master thesis]. If the Standard Model is seen as a low-mass approximation of a more general theory valid uo to the Planck mass scale (i.e., $\approx 10^{19}$ GeV), a fine-tuning cancellation of the order of $1$ over $10^{34}$ is needed in order to protect the Higgs mass at the electroweak scale ($\approx 100$ GeV). Such an astonishing correction is perceived as very unnatural.
%\item {\itshape Problema della scala delle masse.}\\ Perch\'e nel Modello Standard le masse delle particelle fermioniche vanno dall'ordine dell'eV per i neutrini (limite forse molto sovrastimato, come emerge dalle differenze di massa nelle oscillazioni) alla centinaia di GeV del quark top?
\end{itemize}

Numerous Beyond Standard Model theories (BSM) have been proposed in order to overcome the limits of the Standard Model.

Grand Unified Theories (GUT) aim at extending the symmetry group of the SM (eq.~\ref{eq:theory_SMgroup}) into largest candidates, such as $S0(10)$, $SU(5)$ and $E(6)$. At GUT scale, approximately at $10^{16}$ GeV, non-gravitational interactions are expected to be ruled by only one coupling constant, $\alpha_{GUT}$. %Pur godendo di alcuni punti di forza, quali la possibilit\`a di dare massa ai neutrini ammettendo l'esistenza del neutrino right alla scala GUT, queste teorie prevedono in alcuni casi fenomeni non ancora osservati (decadimento del protone, esistenza dei monopoli magnetici) e sono valide ad energie non riproducibili sperimentalmente. Qualche speranza di trovare nuova fisica a scale inferiori potrebbe nascere dal connubio tra GUT e modelli supersimmetrici.

Super Symmetryc (SUSY) models state that every fermion (boson) of the Standard Model has a bosonic (fermionic) superpartner, with exactly the same quantum numbers, except the spin. If SUSY is not broken, each couple of partners and superpartners should have the same masses, hypotesis easily excluded by the non-observation of the s-electron. Super Symmetry represents a very elegant solution of the hierarchy problem of the Higgs boson mass, since the perturbative corrections brought by new SUSY particles exactly cancel out the divergences caused by SM particles corrections. A particular sub-class of SUSY models, Minimal Super Symmetric Standard Models, is characterized by the introduction of a new symmetry, the R-parity, that guarantees the proton stability and also the stability of the lightest SUSY particle, a possible good candidate for dark matter.

Two other possible theoretical pictures are extensively described in sec.~\ref{sec:theory_HVT}-\ref{sec:theory_WED}.


\section{Heavy Vector Triplet}
\label{sec:theory_HVT}

The heavy vector triplet model~\cite{Pappadopulo2014} provides a general framework aimed at studying new physics beyond the standard model, that can manifest into the appearance of new resonances.\\
The adopted approach is that of the simplified model, in which an effective Lagrangian is introduced, in order to describe the properties and interactions of new particles (in this case, a triplet of spin-1 bosons) by using a limited set of parameters, that can be easily linked to the physical observables at the LHC experiments. These parameters can describe many physical motivated theories (such as sequential extensions of the SM~\cite{Barger:1980ix}-\cite{Grojean:2011vu} or Composite Higgs~\cite{Contino2011}-\cite{Bellazzini:2014yua}). \\
Since a simplified model is not a complete theory, its validity is restricted to the on-shell quantities related to the production and decay mechanisms of new resonances, that is the case of most of the LHC BSM searches are performed. Given these conditions, experimental results in the resonant region are sensitive to a limited number of the phenomenological Lagrangian parameters (or to a combination of those), whilst the remaining parameters tend to influence the tail of the distributions.\\
Limits on production cross-section times branching ratio ($\sigma  \mathcal{B}$), as a function of the invariant mass spectrum of the probed resonance, can be extracted from experimental data. Given that $\sigma  \mathcal{B}$ are functions of the simplified model parameters and of the parton luminosities, it is then possible to interpret the observed limits in the parameter space.


The heavy vector triplet framework assumes the existence of an additional vector triplet, $V_{\mu}^a$, $a=1,2,3$, in which two spin-1 particles are charged and one is neutral:

\begin{equation}
V_{\mu}^{\pm} = \frac{V_{\mu}^1 \mp i V_{\mu}^2}{\sqrt{2}}; \mbox{ } V_{\mu}^0 = V_{\mu}^3.
\label{eq:V_triplet}
\end{equation}
\\
The triplet interactions are described by a simplified Lagrangian, that is invariant under SM gauge and CP symmetry, and accidentally invariant under the custodial symmetry $SU(2)_L \times SU(2)_R$:
\begin{equation}
\begin{split}
\mathcal{L_V} ={} & -\frac{1}{4} \left(D_{\mu}V_{\nu}^a - D_{\nu}V_{\mu}^a \right) \left( D^{\mu}V^{\nu \mbox{ } a} - D^{\nu}V^{\mu \mbox{ } a} \right) + \frac{m_V^2}{2}V_{\mu}^aV^{\mu \mbox{ } a} \\
 & +i g_V c_H V_{\mu}^a \left( H^{\dagger} \tau^a D^{\mu}H - D^{\mu}H^{\dagger} \tau^a H \right) + \frac{g^2}{g_V}c_F V_{\mu}^a \sum_{f} \bar{f}_L \gamma^{\mu} \tau^a f_L \\
 & + \frac{g_V}{2} c_{VVV} \epsilon_{abc} V_{\mu}^a V_{\nu}^b \left( D^{\mu}V^{\nu \mbox{ } c} - D^{\nu}V^{\mu \mbox{ } c}\right) + g_v^2 c_{VVHH} V_{\mu}^a V^{\mu \mbox{ } a} H^{\dagger} H - \frac{g}{2} c_{VVW} \epsilon_{abc} W^{\mu \nu \mbox{ } a} V_{\mu}^b V_{\nu}^c.
\end{split}
\label{eq:Lagrangian}
\end{equation}
\\
In the first line of the formula~\ref{eq:Lagrangian}, $V$ mass and kinematic terms are included, described with the covariant derivative $D_{\mu} V_{\nu}^a = \partial_{\mu} V_{\nu}^a + g \epsilon^{abc} W_{\mu}^b V_{\nu}^c$, where $W_{\mu}^a$ are the fields of the weak interaction and $g$ is the weak gauge coupling.\\
The second line describes the interaction of the triplet with the Higgs field and the SM left-handed fermions; $c_H$ describes the vertices with the physical Higgs and the three unphysical Goldstone bosons that, for the Goldstone equivalence theorem, are connected to the longitudinal polarization of W and Z bosons at high-energy; hence, $c_H$ is related to the bosonic decays of the resonances. $c_F$ is the analogous parameter describing the $V$ interaction with fermions, that can be generalized as a flavour dependent coefficent, once defined $J_F^{\mu \mbox{ } a} = \sum_{f} \bar{f}_L \gamma^{\mu} \tau^a f_L$: $c_F V_{\mu}^a  J_F^{\mu \mbox{ } a} = c_{\ell} V_{\mu}^a  J_{\ell}^{\mu \mbox{ } a} + c_{q} V_{\mu}^a  J_{q}^{\mu \mbox{ } a} + c_{3} V_{\mu}^a  J_{3}^{\mu \mbox{ } a}$.\\
{\color{red} \bf CHE COSA C'ENTRA IL GOLDSTONE BOSON EQUIVALENCE THEOREM CON LE W-Z AD ALTA ENERGIA? APPROFONDISCI\\}
The last part of the equation does not describe vertices of $V$ with SM fields, hence it can be neglected while describing the LHC phenomenology, under the assumptions previously made {(\color{red} \bf NEXT SECTION...)}, along with dimension four quadrilinear $V$ interactions, otherwise their effects would be appreciated in electroweak precision tests and precise Higgs coupling measurements.\\

%Comments on the parameter range

In Eq. (2.2) we adopted a rather peculiar parametrization of the interaction terms, with a
coupling g V weighting extra insertions of V , of H and of the fermionic fields. Similarly, the
insertions of W in the last line is weighted by the SU (2) L coupling g. We take g V to represent
the typical strength of V interactions while the dimensionless coefficients “c” parametrize the
departures from the typical size. The parametrization of the fermion couplings is an exception
to this rule. In this case one extra factor of g 2 /g V 2 has been introduced. This is convenient
because in all the explicit models we will be interested in, both of weakly- and strongly-coupled
origin, this factor is indeed present and the c F ’s, as defined in Eq. (2.2), are of order one. The
other c’s are typically of order one, except for c H which is of order one in the strongly-coupled
scenario but can be reduced in the weakly coupled case as described in Section 4. In all cases,
the c’s are never parametrically larger than one, with the notable exception of the third family
coupling c 3 , which could be enhanced in strongly-coupled scenarios where the top quark mass
is realized by the mechanism of Partial Compositeness, see for instance [71]. The coupling g V
can easily vary over one order of magnitude in different scenarios, ranging from g V ∼ g ∼ 1
in the “typical” weakly-coupled case up to g V ' 4π in the extreme strong limit. Therefore
it is useful to factor it out of the operator estimate. Notice that there is no sharp separation
between the weak and strong coupling regimes as nothing forbids to consider theories with a
“weak” UV origin but with large g V , of the order of a few, and “strong” models where g V is
reduced by the large number of colors of the strong sector, g V = 4π/ N c . This provides one
additional motivation for our approach which interpolates between the two cases.

%Comments on the L symmetry


%%%%%%%%%%%%%%%%%%%%%%%%%
%Firstly, this allows us to derive robust model-independent phenomenological features
%and, conversely, to identify the peculiarities of different explicit realizations.

%(The experimental results should be presented in the parameter space of the
%phenomenological Lagrangian, expressed by confidence level curves or, if possible, in terms of
%a likelihood function. In this way they could be easily translated into any specific model where
%the phenomenological parameters can be computed explicitly.)

{\color{red}
the simple but well-motivated example of electroweak-charged spin one resonances which are a
common prediction of many New Physics scenarios. The latter can be weakly coupled, like for
instance Z 0 [6, 17–27] or W 0 [7, 8, 10, 11, 28–32] models, or strongly coupled constructions such
as Composite Higgs models [33–39] and some variants of Technicolor [40–48].}



\subsection{Model A}
\label{sec:theory_HVT_A}

\subsection{Model B}
\label{sec:theory_HVT_B}

\section{Warped extra dimension}
\label{sec:theory_WED}


\clearpage

