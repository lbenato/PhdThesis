\chapter*{Abstract}
\label{ch:abstract}

This thesis presents a search for potential signals of new heavy resonances decaying into a pair of vector bosons, with masses between 1 \TeV and 4 \TeV, predicted by beyond standard model theories. The signals probed are spin-1 \Wp, predicted by the Heavy Vector Triplet model, and spin-2 bulk gravitons, predicted by warped extra-dimension models. The scrutinized data are produced by LHC proton-proton collisions at a center-of-mass energy $\sqrt{s}=13$ \TeV during the 2016 operations, and collected by the CMS experiment, corresponding to an integrated luminosity of 35.9 \fbinv. One of the boson should be a \Z, and it is identified through its invisible decay into neutrinos, while the other electroweak boson, consisting either into a \W or into a \Z boson, is required to decay hadronically into a pair of quarks. The decay products of heavy resonances are produced with large Lorentz boosts; as a consequence, the decay products of the bosons (quarks and neutrinos) are expected to be highly energetic and collimated. The couple of neutrinos, escaping undetected, is reconstructed as missing momentum in the transverse plane of the CMS detector. The couple of quarks is reconstructed as one large-cone jet, with high transverse momentum, recoiling against the couple of neutrinos. Grooming algorithms are adopted in order to improve the jet mass resolution, by removing soft radiation components and spectator events from the particles clustered as the large-cone jet. The groomed jet mass is used to tag the hadronically decaying vector boson, to define the signal region of the search (close to the nominal mass of the \W and \Z bosons, between 65-105 \GeV) and a signal-depleted control region, that is used for the background estimation. An hybrid data-simulation approach predicts the normalization and the shape of the main background, represented by a vector boson produced in association with jets, by taking advantage of the distribution of data in the signal-depleted control regions. Secondary backgrounds are predicted from simulations. Jet substructure techniques are exploited, in order to classify events into two exclusive purity categories, by distinguishing the couple of quarks inside the large-cone jet. This approach improves the background rejection and the discovery reach. The search is performed by scanning the distribution of the reconstructed mass of the resonance, looking for a local excess in data with regards to the prediction. Depending on the mass, upper limits on the cross-section of heavy spin-1 and spin-2 narrow resonances, multiplied by the branching fraction of the resonance decaying into \Z and a \W boson for a spin-1 signal, and into a pair of \Z bosons for spin-2, are set in the range $0.9$ -- $63$ \fb and in the range $0.5$ -- $40$ \fb respectively. A \Wp hypotesis is excluded up to 3.11 \TeV, in the Heavy Vector Triplet benchmark A scenario, and up to 3.41 \TeV, considering the benchmark B scenario. A bulk graviton hypotesis, given the curvature parameter of the extra-dimension $\tilde{k}=1.0$, is excluded up to 1.14 \TeV.


\clearpage
