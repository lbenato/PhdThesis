\chapter{Theoretical motivation}
The Standard Model (SM) of particles represents, so far, the best available description of the particles and their interactions. It is the summation of two gauge theories: the electroweak interaction, that pictures together the weak and electromagnetic interactions, and the strog interaction, or Quantum Chromodynamics (QCD). Particles, namely quarks and leptons, are described by spin 1/2 fermions, whilst interactions are embodied by spin 1 bosons. The simmetry group of the standard model is:
\begin{equation}
SU_{C}(3) \times SU_L (2) \times U_Y (1),
\end{equation}
where the first factor is related to strong interactions, whose mediators are eight gluons, whilst $SU_L (2) \times U_Y (1)$ is the electroweak simmetry group, whose mediators are photons and $Z$-$W^{\pm}$ bosons.

In renormalizable theories, with no anomalies, all gauge bosons are expected to be massless, in contrast with our experimental knowledge (cite W-Z discovery). This kind of dilemma can be solved by introducing a new scalar particle, the Higgs boson (cite Higgs article), that can give mass to weak bosons and fermions via the spontaneous symmetry breaking mechanism. 

In the last decades, Standard Model has been accurately probed by many experimental facilities (LEP, Tevatron), demonstrating an impressive agreement between theoretical predictions and experimental results. The discovery of the Higgs boson at the Large Hadron Collider, measured by both CMS and ATLAS collaborations (cite discovery), represents not only an extraordinary confirmation of the model, but also the latest biggest achievement in particle physics as a whole.

\section{Beyond Standard Model theories}
Even though the Standard Model is the most complete picture of the universe of the particles, many questions are still left open by the model. From a phenomenological point of view, some experimental observations are not included in the theory:
\begin{itemize}
\item in SM, neutrinos are massless (whilst experimentally it has been confirmed to be non-zero, i.e. by the neutrino oscillation);
\item no candidate for the dark matter is foreseen (whilst it has been observed in cosmology);
\item no fields included in the SM can explain the cosmological inflation;
\item SM can not justify the matter-antimatter asymmetry.
\end{itemize}
From a purely theoretical perspective, some issues are still relevant in the formulation of the model:
\begin{itemize}
\item {\itshape Flavour problem.}\\ The Standard Model has 18 free parameters: 9 fermionic masses; 3 angular parameters in Cabibbo-Kobayashi-Maskawa matrix, plus 1 phase parameter; electromagnetic coupling $\alpha$; strong coupling $\alpha_{strong}$;  weak coupling $\alpha_{weak}$; $Z$ mass; the mass of the Higgs boson. Such a huge number of degrees of freedom is considered as weakly predictive.
\item {\itshape Unification.}\\ There is not a ``complete'' unification of strong, weak and electromagnetic interactions, since each one has its own coupling constant, behaving differently at different energy scales; not to mention the fact that gravitational interaction, completely excluded from the SM.
\item {\itshape Hierarchy problem.}\\ From Quantum Field Theory, it is known that perturbative corrections to the mass of the scalar bosons included in the theory tend to make it increase towards the energy scale at which the considered theory is valid [cite n. 4 of master thesis]. If the Standard Model is seen as a low-mass approximation of a more general theory valid uo to the Planck mass scale (i.e., $\approx 10^{19}$ GeV), a fine-tuning cancellation of the order of $1$ over $10^{34}$ is needed in order to protect the Higgs mass at the electroweak scale ($\approx 100$ GeV). Such an astonishing correction is perceived as very unnatural.
%\item {\itshape Problema della scala delle masse.}\\ Perch\'e nel Modello Standard le masse delle particelle fermioniche vanno dall'ordine dell'eV per i neutrini (limite forse molto sovrastimato, come emerge dalle differenze di massa nelle oscillazioni) alla centinaia di GeV del quark top?
\end{itemize}

Numerous Beyond Standard Model theories (BSM) have been proposed in order to overcome the limits of the Standard Model.
\subsection{Warped extra dimension}
\subsection{Heavy Vector Triplet}

\clearpage

