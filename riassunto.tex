\chapter*{Riassunto}
\label{ch:riassunto}

Questa tesi presenta una ricerca di potenziali segnali di nuove risonanze pesanti, che decadono in una coppia di bosoni vettori, con masse comprese tra 1 \TeV e 4 \TeV, predette da teorie oltre il modello standard. I segnali indagati sono \Wp di spin 1, predette dal modello Heavy Vector Triplet, e gravitoni di spin 2, predetti da modelli che prevedono extra dimensioni ripiegate. I dati esaminati sono prodotti dalle collisioni protone-protone di LHC ad un'energia del centro di massa di $\sqrt{s}=13$ \TeV durante le operazioni del 2016, e raccolti dall'esperimento CMS, per una luminosit\`a integrata di 35.9 \fbinv. Uno dei bosoni dev'essere una \Z, che viene identificata dal suo decadimento invisibile in neutrini ($\nu \bar{\nu}$), mentre l'altro bosone elettrodebole, sia una \W che una \Z, deve decadere nel canale adronico in una coppia di quark ($q \bar{q}$). I prodotti di decadimento di risonanze pesanti sono prodotti con significativi boost di Lorentz; di conseguenza, i prodotti di decadimento dei bosoni (i quark e i neutrini) sono attesi avere elevate energie ed essere collimati. La coppia di neutrini, che sfugge alla rivelazione, viene ricostruita come momento mancante nel piano trasverso del rivelatore CMS. La coppia di quark viene ricostruita come un jet a largo cono, con elevato momento trasverso, che rincula contro la coppia di neutrini. Algoritmi di grooming sono impiegati per migliorare la risoluzione della massa del jet, rimuovendo la radiazione soffice e gli eventi spettatori dalle particelle clusterizzate come jet a largo cono. La massa ripulita del jet viene utilizzata per identificare il bosone vettore che decade in adroni, per definire la regione di segnale della ricerca (vicina alla massa nominale dei bosono \W e \Z, nell'intervallo 65-105 \GeV) e una regione di controllo svuotata dal segnale, che viene utilizzata per la stima dei fondi. Un approccio ibdrido dati-simulazione predice la normalizzazione e la forma del fondo principale, rappresentato da un bosone vettore prodotto in associazione con jet, sfruttando la distribuzione dei dati nelle regioni di controllo svuotate dal segnale. I fondi secondari sono predetti completamente con le simulazioni. Tecniche di sottostruttura del jet sono adoperate per classificare gli eventi in due categorie esclusive di purezza, distinguendo le coppie di quark dentro al jet a largo cono. Questo approccio migliora la soppressione del fondo e la potenzialit\`a di scoperta. La ricerca viene fatta scansionando la distribuzione della massa ricostruita della risonanza, cercando in eccesso locale nei dati rispetto alle predizioni. In funzione della massa, limiti superiori sulla sezione d'urto per risonanze pesanti e strette di spin 1 e spin 2, moltiplicate per il rapporto di diramazione della risonanza che decade in \Z e \W per il segnale di spin 1, e in una coppia di bosoni \Z per lo spin 2, sono fissati nell'intervallo $0.9$ -- $63$ \fb e nell'intervallo $0.5$ -- $40$ \fb rispettivamente. Un'ipotesi di \`e esclusa fino ad una massa di 3.11 \TeV, nello scenario A di riferimento dell'Heavy Vector Triplet, e fino a 3.41 \TeV, nello scenario B di riferimento. Un'ipotesi di gravitone, dato il parametro di curvatura della dimensione addizionale $\tilde{k}=1.0$, \`e esclusa fino ad una massa di 1.1 \TeV.

\clearpage

