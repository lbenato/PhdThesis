\chapter{Conclusions}
\label{ch:conclusion}

This thesis presented a search for heavy resonances with masses between 1 \TeV and 4 \TeV, decaying into a pair of vector bosons, predicted by beyond standard model theories. The data produced by LHC proton-proton collisions, at a center-of-mass energy $\sqrt{s}=13$ \TeV during the 2016 operations, and collected by the CMS experiment, corresponding to an integrated luminosity of 35.9 \fbinv, are analyzed. The probed final state includes the invisible decay modes of one $Z$ boson, reconstructed as a large amount of missing transverse momentum, and the hadronic decay of the other vector boson (\Z, \W), reconstructed as a large-cone jet. The collected events are divided into two purity categories, based on the substructure of the hadronically decaying \V boson. No significant excesses over the expected background are observed in the entire mass range probed by the analysis.

\noindent Depending on the resonance mass, 95\% C.L. upper limits on the cross-section of heavy spin-1 and spin-2 narrow resonances, multiplied by the branching fraction of the resonance decaying into \Z and a \W boson for a spin-1 signal, and into a pair of \Z bosons for spin-2, are set in the range $0.9$ -- $63$ \fb and in the range $0.5$ -- $40$ \fb respectively. A \Wp hypotesis is excluded up to 3.11 \TeV, in the context of the Heavy Vector Triplet model A scenario, and up to 3.41 \TeV, considering the model B scenario. A bulk graviton hypotesis, given the curvature parameter $\tilde{k}=1.0$, is excluded up to 1.14 \TeV.

\vspace*{1\baselineskip}

\noindent This is the first search for $\VZ \rightarrow q \bar{q} \nu \bar{\nu}$ performed by the CMS Collaboration at $\sqrt{s} = $ 13 \TeV. This analysis is part of a set of searches for heavy resonances decaying into dibosons. The future perspectives of the analysis consist both in the combination of this final state with other diboson searches sharing the same treatment of one boson hadronic decay (namely, the same definition of the sidebands and signal regions), and in the combination of the 2016 data with the newly collected 2017 data. The luminosity planned to be delivered by the LHC collider in 2017 is comparable to what was collected in 2016 ($\sim 40$ \fbinv). By doubling the statistics, marginal improvements are foreseen; hence, a larger enhancement can be achieved by decreasing the impacts of the systematic uncertainties. New ideas are currently being tested, in order to improve the jet mass resolution (recursive soft drop), suppress the pile-up contribution (PUPPI associated to SoftKiller algorithm~\cite{Cacciari:2014gra}), exploit the jet substructure and tag the nature of a large-cone jet (originating from \W, \Z, Higgs boson or top quark) with machine learning techniques; preliminary results on these new methods seem to be promising.

\clearpage

