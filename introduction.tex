\chapter{Introduction}

The discovery of the Higgs boson at the CERN Large Hadron Collider represents a milestone in the knowledge of the particle physics. The Higgs mechanism connects the theoretical formulation of the standard model of the particles to the current picture of the universe, as it is known: spin-1 weak bosons and standard model fermions are allowed to acquire masses, constituting the known matter. Despite this successfull achievement, some questions are still left unanswered; in order to solve the open problems, a plethora of new beyond standard model theories have been built.

\noindent Many of these theories predict the existence of larger symmetries in the universe, or new extra-dimensions, that will result into the appearance of new heavy particles, expected to have a mass around 1 \TeV. The Large Hadron Collider (LHC) is the ideal tool to investigate this unknown phase-space, given the fact that during the so-called LHC Run 2 era (started in 2015), the unprecedented center-of-mass energy of 13 \TeV has been reached in the proton-proton collisions.

\noindent The CMS experiment, located in the northern part of the LHC ring, is a multi purpose detector, suitable to study highly energetic new phenomena. Its intense magnetic field, its sharp segmentation, its hermeticity and the interplay of many sophisticated reconstruction algorithms allow to measure with a very high precision the trajectories, the momenta and the energy deposits left by energetic particles.

\vspace*{1\baselineskip}

\noindent This thesis presents a searche for signal of heavy resonances that decay into a pair of vector bosons. The search is performed by used the 2016 data produced by proton-proton collisions of the LHC, and collected by the CMS detector. One $Z$ boson is identified through its invisible decay in neutrinos, while the other vector boson is required to decay hadronically into a pair of quarks. Given the fact that the searched resonances have masses around the \TeV, their decay products are expected to be produced with large Lorentz boosts. This leads to a non-trivial identification of the couple of quarks or leptons, coming from the vector bosons decays. In fact, they are expected to lay very close in angle. Dedicated algorithms and substructure techniques allow to distinguish a pair of quarks originating from a vector boson from the background processes, initiated by the strong interaction.

\noindent The search is performed by scanning the distribution of the reconstructed mass of the resonance, looking for a signal local excess in data with regards to the prediction. The background prediction is performed with an hybrid data-simulation approach, by using the distribution of data in signal-depleted control regions.

\newpage

\noindent The thesis is organized as follows.

\noindent In chapter~\ref{chap:theory}, an overwiev of the theoretical motivations is presented. Two beyond standard model theories are particularly appealing: the Heavy Vector Triplet Model and the bulk graviton model.

\noindent In chapter~\ref{chap:LHC_CMS}, the CMS detector is briefly described, along with the physics objects exploited for the purpose of this search.

\noindent Chapter~\ref{ch:analysis} is dedicated to the analysis: after a general introduction (sec.~\ref{sec:analysis_overview}), the features of the data, signal and background samples used in the analysis are described in detail (sec.~\ref{sec:samples}). Sec.~\ref{sec:objects} is dedicated to the selections applied, in order to reach the best signal-to-noise efficiency and to properly build the resonance candidate. The very first data-simulation comparison is performed in sec.~\ref{sec:datamc_comp}. The background estimation technique, the final data-predicted background comparison and the signal modelling are included in sec.~\ref{sec:alpha}. Systematic uncertainties are listed in sec.~\ref{sec:uncertainties}. The final results, the statistical analysis and the physics interpretation are shown in sec.~\ref{sec:results}. Chapter~\ref{ch:conclusion} summarizes the conclusions.

\clearpage

